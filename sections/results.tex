% 结果部分(占位数据使用 \TODO 标记)

\makeatletter
\@ifundefined{c@todoctr}{\newcounter{todoctr}}{}
\makeatother
\providecommand{\TODO}{%
  \stepcounter{todoctr}%
  {\small\textbf{\textit{\textsf{*\thetodoctr}}}}}
\setcounter{todoctr}{0}

\section{结果}

\subsection{客观性能评估结果}

\subsubsection{回顾性测试性能表现}

\paragraph{信息提取准确度}

针对 \TODO 份回顾性测试病历的 18 个核心槽位评估显示,系统整体精确率达 \TODO,召回率达 \TODO,F1 分数达 \TODO;其中“初步诊断”“治疗计划”等关键槽位 F1 分数均超过 \TODO,种植体型号、牙槽骨骨量等专科高价值信息提取精确率与召回率均达 \TODO 以上,满足“专科适配性达标”要求。

\paragraph{信息覆盖率}

系统生成病历的字段级覆盖率达 \TODO,整体内容覆盖率达 \TODO,核心诊疗槽位内容覆盖率均 $\geq \TODO$,信息覆盖水平判定为“优秀”。

\paragraph{文本相似度与规范性}

生成病历与金标准病历的 BERTScore 均值达 \TODO,ROUGE-L 均值达 \TODO;经术语库优化后,文本术语规范率达 \TODO,较未优化前提升 \TODO,有效降低了 ASR 识别误差对病历质量的影响。多提示词与多评分者下的精确率、召回率与 F1 分数分布如图~\ref{fig:multi_prompt_performance} 所示,文本相似度与规范性指标的对比见图~\ref{fig:text_similarity}。

\subsubsection{前瞻性验证性能表现}

\paragraph{效率指标}

试验组平均病例录入耗时 \TODO~min/例 显著短于对照组 \TODO~min/例($P<0.001$);试验组日均完成病历数量为 \TODO 份,显著高于对照组的 \TODO 份($P<0.01$)。不同临床场景下人工书写与系统辅助书写的时间分布比较见图~\ref{fig:efficiency_violin},展示了整体、急诊、单科会诊和多科会诊等场景中系统带来的效率提升。

\paragraph{质量指标}

试验组病例完整性评分(均值 $\pm$ 标准差)为 \TODO 分,高于对照组的 \TODO 分($P<0.001$);两组信息准确率无统计学差异($P>0.05$);试验组诊疗逻辑一致性达标率为 \TODO,与对照组的 \TODO 相当($P>0.05$)。

\paragraph{主观与安全指标}

试验组用户满意度评分为 \TODO 分,显著高于对照组的 \TODO 分($P<0.001$);试验期间未发生严重数据泄露或系统故障等不良事件。

\subsection{主观质量评估结果}

\subsubsection{临床专家盲评得分}

\paragraph{综合质量表现}

\TODO 名资深临床专家(副主任医师及以上)的盲评结果显示,生成病历在 Likert 量表 7 个评估维度的平均得分均超过 \TODO 分(5 分制)。其中结构规范性(\TODO 分)和术语准确性(\TODO 分)得分最高,信息完整性(\TODO 分)和逻辑一致性(\TODO 分)表现优异,临床实用性(\TODO 分)和可读性(\TODO 分)均满足临床应用需求,安全性(\TODO 分)评分证实系统无隐私信息泄露风险。

\paragraph{专科专家评价差异}

不同专科专家对本专科场景生成病历的评分无显著差异($P>0.05$),且均高于对通用场景病历的评分,表明模板库与专科优化策略适配临床实际需求。医生偏好与多维度质量对比结果汇总于图~\ref{fig:preference_risk},其中包括总体偏好量表、不同任务得分分布、读者偏好比例以及风险评估等模块;图~\ref{fig:likert_radar} 展示主诉、既往史、检查、诊断和治疗计划等关键模块上的评分差异,为主观质量章节提供更细粒度的可视化证据。

\subsubsection{幻觉现象专项评估}

在 \TODO 份测试病历中,仅出现 \TODO 例幻觉现象,幻觉发生率为 \TODO。对比实验显示,通用生成式 LLM 的幻觉发生率为 \TODO,本系统通过“信息提取 + 知识库校验”的设计,显著抑制了幻觉风险($P<0.001$)。

\subsection{消融实验结果}

\subsubsection{知识库纠错模块的作用}

移除知识库纠错模块后,系统信息提取的精确率降至 \TODO,较完整系统下降 \TODO 个百分点;关键信息错误率(如药品名称错误、术语不规范)从 \TODO 升至 \TODO,证实知识库校验能有效提升信息准确性。

\subsubsection{信息提取范式的优势}

将 LLM 改为自由生成模式后,幻觉发生率升至 \TODO,较原设计(\TODO)显著升高($P<0.001$);文本相似度指标中,BERTScore 降至 \TODO,ROUGE-L 降至 \TODO,且病历结构规范性评分下降至 \TODO 分,表明限定 LLM 为“信息提取器”的设计能同时保障安全性、准确性与规范性。知识库纠错模块和信息提取范式的综合效果对比如图~\ref{fig:ablation} 所示。

\subsection{迭代优化效果验证}

经过 \TODO 个月的临床试用与数据迭代,系统回收并脱敏处理 \TODO 份有效病历数据,用于更新错词库、术语库及模型微调。迭代后,ASR 专科术语识别准确率提升 \TODO 个百分点,LLM 信息提取 F1 分数提升 \TODO 个百分点,医生审核修改率从 \TODO 降至 \TODO,实现了系统性能的持续优化,验证了闭环验证框架的有效性。

